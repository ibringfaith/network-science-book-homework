\documentclass{article}%
\usepackage[T1]{fontenc}%
\usepackage[utf8]{inputenc}%
\usepackage{lmodern}%
\usepackage{textcomp}%
\usepackage{lastpage}%
\usepackage{geometry}%
\geometry{tmargin=1cm,lmargin=10cm}%
\usepackage{amsmath}%
%
%
%
\begin{document}%
\normalsize%
\section{Use the rate equation approach to show that the directed copying model leads to a scale{-}free network with incoming degree exponent gamma\_i=(2{-}p)/(1{-}p).}%
\label{sec:Usetherateequationapproachtoshowthatthedirectedcopyingmodelleadstoascale{-}freenetworkwithincomingdegreeexponentgammai=(2{-}p)/(1{-}p).}%
Let us denote with N(k,t) the number of nodes with degree k at time t and the degree distribution%
\begin{alignat*}{2}%
p_k(t)&=\frac{N(k,t)}{N(t)}\\%
\end{alignat*}%
The expected change in N(k,t) after one timestep is%
\begin{alignat*}{2}%
N(k, t+1) - N(k, t))\\%
= p[(k-1)\frac{N(k-1,t)}{N(t)}-k\frac{N(k,t)}{N(t)}] + (1-p)(\frac{N(k-1,t)}{N(t)})\\%
= p[\frac{(k-1)N(k-1,t)-kN(k,t)}{t}] + (1-p)(\frac{N(k-1, t)-N(k,t)}{t})\\%
\end{alignat*}%
When t is large, we assume that N(k,t)=t*p\_k such that the equation now becomes%
\begin{alignat*}{2}%
p_k&=p[(k-1)p_{k-1}-kp_k] + (1-p)[p_k-p_{k-1}]\\%
\end{alignat*}%
Rewriting the equation we get%
\begin{alignat*}{2}%
p_k&=p_{k-1}\cdot\frac{k-1+(1/p)}{k+1+(1/p)}\\%
\end{alignat*}%
Using a recursive approach we see that%
\begin{alignat*}{2}%
p_k&=C\prod_{j=i}^{k}\frac{j-1+(1/p)}{j+1+(1/p)}\\%
\end{alignat*}%
which is about%
\begin{alignat*}{2}%
k^{\frac{-(2-p)}{(1-p)}}\\%
\end{alignat*}%
when k is sufficiently large.

%
\end{document}
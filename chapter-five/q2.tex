\documentclass{article}%
\usepackage[T1]{fontenc}%
\usepackage[utf8]{inputenc}%
\usepackage{lmodern}%
\usepackage{textcomp}%
\usepackage{lastpage}%
\usepackage{geometry}%
\geometry{tmargin=1cm,lmargin=10cm}%
\usepackage{amsmath}%
%
%
%
\begin{document}%
\normalsize%
\section{Calculate, using the rate equation approach, the in{-} and out{-} degree distribution of the resulting network.}%
\label{sec:Calculate,usingtherateequationapproach,thein{-}andout{-}degreedistributionoftheresultingnetwork.}%
Let us denote with N(k,t) the number of nodes with degree k at time t. Then the rate equation is%
\begin{alignat*}{2}%
m[\prod(k-1)N(k-1,t)-\prod(k)N(k,t)]\\%
\end{alignat*}%
Substitutiong in the given probability, we can simplify the rate equation to%
\begin{alignat*}{2}%
m[N(k-1,t)\cdot\frac{(k-1)+A}{mt+At}-N(k,t)\cdot\frac{k+A}{mt+At}]\\%
\end{alignat*}%
where mt is the sum of all the in{-}degrees.%
When t is large, we assume that N(k,t)=t*p\_k such that%
\begin{alignat*}{2}%
p_k&=m[p_{k-1}\cdot\frac{(k-1)+A}{m+A}-p_k\cdot\frac{k+A}{m+A}]\\%
\end{alignat*}%
which simplifies to%
\begin{alignat*}{2}%
p_k&=p_{k-1}\cdot\frac{m(k-1+A)}{k+m+2A}\\%
\end{alignat*}

%
\section{By using the properties of the Gamma and Beta functions, can you find a power{-}law scaling for the in{-}degree distribution?}%
\label{sec:ByusingthepropertiesoftheGammaandBetafunctions,canyoufindapower{-}lawscalingforthein{-}degreedistribution?}%
Using a recursive approach we see that%
\begin{alignat*}{2}%
p_k&=p_1\prod_{j=1}^{k-1}\frac{m(j+A)}{j+1+m+2A}\\%
\end{alignat*}%
Using the properties of the Gamma and Beta functions, we get%
\begin{alignat*}{2}%
p_k&=k^{-2+\frac{A}{m+A}}\\%
\end{alignat*}%
when k is sufficiently large.

%
\section{For A = 0 the scaling exponent of the in{-}degree distribution is different from gamma = 3, the exponent of the Barabási{-}Albert model. Why?}%
\label{sec:ForA=0thescalingexponentofthein{-}degreedistributionisdifferentfromgamma=3,theexponentoftheBarabsi{-}Albertmodel.Why?}%
The Barabási{-}Albert model is undirected so whe attaching new nodes to the graph, it considers the total number of degrees in the graph rather than just the in{-}degrees. This affects the degree distribution, and as a result, the scaling exponent.

%
\end{document}
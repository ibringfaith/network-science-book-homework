\documentclass{article}%
\usepackage[T1]{fontenc}%
\usepackage[utf8]{inputenc}%
\usepackage{lmodern}%
\usepackage{textcomp}%
\usepackage{lastpage}%
\usepackage{geometry}%
\geometry{tmargin=1cm,lmargin=10cm}%
\usepackage{amsmath}%
%
%
%
\begin{document}%
\normalsize%
\section{Derive the time evolution of the node degrees, telling us how many dances a node had.}%
\label{sec:Derivethetimeevolutionofthenodedegrees,tellingushowmanydancesanodehad.}%
Based on the given the probability of node i receiving a dance invitation, the rate of change in the number of dance partners a node has, or the degree of the node, over time is also%
\begin{alignat*}{2}%
\frac{\eta_i}{t<\eta>}\\%
\end{alignat*}%
To solve for the time evolution, we integrate over the amount of time that node is at the party. We add one since there is one already partying node.%
\begin{alignat*}{2}%
k(t, t_i, \eta_i) &= 1+\int_{t_i}^{t}\frac{\eta_i}{t<\eta>}dt\\%
k(t, t_i, \eta_i) &= 1+\frac{\eta_i}{<\eta>}\frac{t}{t_i}\\%
\end{alignat*}

%
\section{Derive the degree distribution of nodes with attractiveness eta.}%
\label{sec:Derivethedegreedistributionofnodeswithattractivenesseta.}%
Solve for the probability that a node with attractiveness eta has degree k.%
\begin{alignat*}{2}%
P(k|<\eta>)&=\frac{d}{dk(t, t_i, \eta_i)}\cdot t_i \cdot P(t_i)\\%
P(k|<\eta>)&=\frac{d}{dk(t, t_i, \eta_i)}\cdot t_i \cdot \frac{1}{t}\\%
P(k|<\eta>)&=\frac{d}{dk(t, t_i, \eta_i)}\cdot te^{-\frac{<\eta>(k(t, t_i, \eta_i)-1)}{\eta_i}} \cdot \frac{1}{t}\\%
P(k|<\eta>)&=\frac{d}{dk(t, t_i, \eta_i)}\cdot e^{-\frac{<\eta>(k(t, t_i, \eta_i)-1)}{\eta_i}}\\%
P(k|<\eta>)&=\frac{\eta}{\eta_i}e^{-\frac{<\eta>(k(t, t_i, \eta_i)-1)}{\eta_i}}\\%
\end{alignat*}

%
\section{If half of the nodes have eta = 2, and the other half eta = 1, what is the degree distribution of the network after a sufficiently long time?}%
\label{sec:Ifhalfofthenodeshaveeta=2,andtheotherhalfeta=1,whatisthedegreedistributionofthenetworkafterasufficientlylongtime?}%
\begin{alignat*}{2}%
\frac{1}{2}\frac{3}{4}e^{-\frac{3(k(t, t_i, \eta_i)-1)}{4}} + \frac{1}{2} \frac{3}{2} \frac{e^{-3(k(t, t_i, \eta_i)-1)}}{2}\\%
\end{alignat*}

%
\end{document}